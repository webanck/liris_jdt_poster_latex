\documentclass[final]{beamer}

% [FR] Vérifiez que votre éditeur encode le fichier source en UTF-8 afin d'éviter
% d'avoir des problèmes.
% [EN] Make sure your editor uses UTF-8 encoding to avoid some issues.
\usepackage[utf8]{inputenc}
% [FR] Pensez à bien déclarer la bonne langue à babel, afin notamment d'avoir un
% espacement correct autour des signes de ponctuation.
% [EN] Remember to tell babel which language you're using to get correct spacing
% around punctuation marks, etc.
\usepackage[francais]{babel}
%\usepackage[english]{babel}
\usepackage[T1]{fontenc}
% Times new roman.
\usepackage{times}

% [FR] Chargement de beamerposter.sty.
% "scale" = Facteur d'agrandissement du texte.
% [EN] Loading beamerposter.sty.
% "scale" = Scaling factor for all text.
\usepackage[
    orientation = portrait,
    size        = a0,
    scale       = 2
]{beamerposter}

% [FR] Chargement de beamerthemelirisposter.sty.
% [EN] Loading beamerthemelirisposter.sty.
\usepackage{beamerthemelirisposter}

% [FR] Si vous n'avez pas besoin de bibliographie, je suppose que vous pouvez
% retirer les packages csquotes et biblatex, mais dans ce cas pensez également à
% retirer les lignes telles que \addbibresource{demo.bib}
% ou encore \renewcommand*{\bibfont}{\small}, et bien sûr les \cite et
% l'appel à \printbibliography situés plus loin dans le document.
% Gestion automatique des guillemets dans les références en prenant en compte
% la langue spécifiée à babel.
% [EN] If you do not need a bibliography, you can remove the csquotes and
% biblatex packages, but if you do, remember to also remove bibliography-related
% lines, such as \addbibresource{demo.bib} or \renewcommand*{\bibfont}{\small},
% and of course the \cite and \printbibliography that come later in the file.
\usepackage{csquotes}
% [FR] Pour une meilleur gestion de l'UTF-8 dans la bibliographie, on peut aussi
% utiliser le programme "biber" et spécifier "backend = biber" à biblatex, mais
% biber n'est pas toujours inclus dans les distributions LaTeX.
% [EN] The "biber" backend program handles UTF-8 better, but as it is not
% always included in LaTeX distributions, I will leave "bibtex" here for now.
\usepackage[backend = bibtex]{biblatex}
\addbibresource{demo.bib}
% [FR] Taille de police pour la bibliographie.
% [EN] Font size for the bibliography.
\renewcommand*{\bibfont}{\small}


\title{Guerre et paix}
\subtitle{Un roman de Léon \bsc{Tolstoï}}
\author{AUTEURS}
\institute{%
    \begin{normalsize}
        Laboratoire d'InfoRmatique en Image et Systèmes d'information
    \end{normalsize}
    
    \begin{scriptsize}
        LIRIS UMR~5205 CNRS\,/\,%
        INSA de Lyon\,/\,%
        Université Claude Bernard Lyon~1\,/\,%
        Université Lumière Lyon~2\,/\,%
        École Centrale de Lyon
    \end{scriptsize}%
}


% [FR] Le poster se compose uniquement d'une "diapositive" ("frame"). Le
% contenu sera centré verticalement, sauf cas très particuliers.
% [EN] The poster is comprised of a single Beamer frame. Its contents will
% generally be vertically centred.
\begin{document}
\begin{frame}

% [FR] Division en deux colonnes, la première occupant 65 % de la largeur
% disponible et la seconde prenant le reste.
% [EN] Creating two columns, with the first one spanning 65 % of available
% width and the other using the rest.
\liriscolumns{65}{
    % [FR] Création d'un bloc en donnant son titre et son contenu.
    % [EN] Creating a block, giving its title and contents.
    \lirisblock{Commandes principales}{
        $\bullet$ \texttt{\textbackslash lirisblock[<Hauteur>]\{<Titre>\}\{<Texte>\}}
        
        Crée un bloc, éventuellement avec une hauteur fixée.
        
        $\bullet$ \texttt{\textbackslash liriscolumns[<Ht>]\{<Larg>\}\{<Gche>\}\{<Drte>\}}
        
        Crée deux colonnes, la première occupant \texttt{<Larg>}\,\% de la largeur disponible et la seconde le reste. Le contenu des colonnes est donné par \texttt{<Gche>} et \texttt{<Drte>}. Si une hauteur \texttt{<Ht>} est spécifiée, les deux colonnes l'adopteront.
        
        $\bullet$ \texttt{\textbackslash lirisinterline}
        
        Crée un espacement vertical correspondant à la distance utilisée entre les colonnes de \texttt{\textbackslash liriscolumns}. Utile pour un rendu uniforme, mais certains préféreront utiliser \texttt{\textbackslash vfill}, afin de répartir le contenu sur tout le poster.
        
        Vous trouverez davantage d'explications en commentaires dans le \texttt{.tex} et dans \texttt{beamerthemelirisposter.sty}.
    } % Fin du bloc. / End of the block.
}{ % Début de la colonne de droite. / Beginning of the right column.
    \lirisblock{Listes imbriquées}{
        \begin{itemize}
            \item La littérature russe
            \begin{itemize}
                \item Fedor Dostoïevski
                \begin{itemize}
                    \item Crime et châtiment
                    \item Le joueur
                    \item Les frères Karamazov
                \end{itemize}
                \item Leo~N. Tolstoï
                \begin{itemize}
                    \item Guerre et paix
                    \item Anna Karénine
                \end{itemize}
            \end{itemize}
            \item La littérature autrichienne
            \begin{itemize}
                \item Elfriede Jelinek
                \item Robert Musil
                \begin{itemize}
                    \item L'homme sans qualités
                    \item Les désarrois de l'élève Törless
                \end{itemize}
            \end{itemize}
        \end{itemize}
    }
}

% [FR] Espacement vertical égal à l'écart entre les colonnes crées
% par \liriscolumns. Sinon, utiliser \vfill pour opter pour un étalement du
% contenu du poster sur toute la hauteur disponible.
% [EN] Vertical spacing matching the length of the gap between the columns
% that \liriscolumns creates. If you prefer to have your elements spread out
% over the poster's height, you might want to use \vfill instead.
\lirisinterline

\liriscolumns{30}{
    % [FR] Ce bloc a une hauteur fixée à 19 cm. La même hauteur sera donnée
    % au bloc se trouvant dans la colonne de droite afin que les deux
    % soient similaires.
    % [EN] This bloc has a fixed height of 19 cm. The same height will be
    % applied to the block that sits in the right column, in order to make
    % them look similar.
    \lirisblock[19cm]{Listes numérotées}{%
        \begin{enumerate}
            \item Villes
            \begin{enumerate}
                \item Lyon
                \item Paris
            \end{enumerate}
            \item Nourriture
            \begin{enumerate}
                \item Tourte
                \item Radis
                \begin{enumerate}
                    \item Blanc
                    \item Noir
                \end{enumerate}
            \end{enumerate}
        \end{enumerate}%
    }
}{
    % [FR] Comme annoncé, bloc avec la même hauteur que celui se trouvant
    % à sa gauche.
    % [EN] As stated, we give the same height to this block.
    \lirisblock[19cm]{Un tableau au milieu d'un bloc de même hauteur}{
        % Pour centrage vertical. / Vertical centring.
        \vfill
        \begin{center}
            % [FR] Les espacements par défaut des tableaux peuvent sembler
            % inadaptés pour les posters. \tabcolsep gère l'espace entre les
            % colonnes, et \arraystretch (un facteur valant 1 par défaut) celui
            % entre les lignes. Voici des exemples de réglages :
            % [EN] Default spacings in tables can seem inappropriate for
            % posters. \tabcolsep defines the space left between columns, and
            % \arraystretch (a factor with a default value of 1) can be used to
            % change the spacing between lines. Here's an example:
            \setlength{\tabcolsep}{0.35em}
            \renewcommand{\arraystretch}{1.25}
            \begin{tabular}{| l | c | c | c |}
                \hline
                Méthode     & Classée~1 & Classée~2 & Classée~3 \\
                \hline
                Méthode~1   & 18        & 10        & 36        \\
                Méthode~2   & 13        & 39        & 12        \\
                Méthode~3   & 33        & 15        & 16        \\
                \hline
            \end{tabular}
        \end{center}
        % Pour centrage vertical. / Vertical centring.
        \vspace*{\fill}
    }
}

\lirisinterline

\lirisblock{Bloc seul sur sa ligne, avec une citation}{
    % [FR] Pour les guillemets français, utiliser :
    %   \og Blabla \fg{}
    % ou
    %   << Blabla >>
    % Babel s'occupe du remplacement, de l'espacement, etc.
    % En anglais (ou en cas d'imbrications de guillemets), cependant, utilisez :
    %   ``Blabla''
    % (Double accent grave pour ouvrir, et double apostrophe pour fermer.)
    % Enfin, notez qu'on met généralement en évidence les mots-clefs
    % avec \alert{blabla}, une commande définie par Beamer.
    % [EN] To obtain French quotation marks, if you're using babel with the
    % "francais" (or "french, or "frenchb") option, you can use these:
    %   \og Text \fg{}
    % or
    %   << Text >>
    % For English quotation marks, use:
    %   ``Text''
    % (Double grave accent ("backtick") to open, double apostrophe to close.)
    % Moreover, note that Beamer's \alert{text} command can be used to
    % highlight keywords.
    << Un logiciel est dit \alert{``libre''} si sa licence donne à chacun le droit d'\alert{utiliser} pour tous usages, d'\alert{étudier}, de \alert{modifier} et de \alert{publier} ces modifications, de \alert{dupliquer} et \alert{donner} ce logiciel. >>~\cite{fs-def}
}

\lirisinterline

% [FR] Deux colonnes de même hauteur.
% [EN] Two columns having the same height.
\liriscolumns[25cm]{50}{
    % [FR] Bloc occupant toute la hauteur de la colonne.
    % [EN] Bloc spanning the full column height.
    \lirisblock[25cm]{Mathématiques, etc.}{
        \begin{description}
            \item[$H_0$] La méthode~3 est aussi efficace que les deux autres
            \item[$H_A$] La méthode~3 est soit plus efficace soit moins efficace qu'une ou que les deux autres méthodes.
        \end{description}
        
        \[
            U \sim B(N, p)
            \qquad
            N = 64,
            \qquad
            p = \frac{1}{3}
        \]
        \[
            P(U = 33)
            =
            \left(
                \begin{array}{c}
                    N   \\
                    33  \\
                \end{array}
            \right)
            \pi^{p}(1 - \pi)^{N - 3}
            \simeq
            0.00111
        \]
    }
}{
    \vfill

    \begin{center}
        % [FR] Exemple de figure afin de montrer qu'on peut mettre du contenu
        % sans forcément le placer dans un \lirisblock. Cela vous montre en même
        % temps quelques principes de TikZ, et une utilisation des couleurs
        % définies par le thème : lirisdarkblue, etc.
        % [EN] A figure showing that content does not necessarily have to be put
        % in a \lirisblock. It also showcases basic TikZ functionalities and
        % uses colours that were defined in the theme.
        \begin{tikzpicture}[
                    scale = 1.25,
                    lirisdarkblue,
                    line width = 15pt,
                    every node/.style = {
                        draw = lirismediumblue,
                        circle,
                        fill = lirislightblue,
                        text = black,
                        font = \Large\bfseries,
                        minimum size = 140pt
                    }
                ]
            \node (n1) at  (0, 0) {1};
            \node (n2) at  (5, 3) {2};
            \node (n3) at (12, 1) {3};
            \draw (n1) -- (n2) -- (n3);
        \end{tikzpicture}
    \end{center}

    % [FR] Grâce au \vfill, le bloc ci-dessous est poussé vers le bas de la
    % colonne, et son extrémité inférieure sera alignée avec celle du bloc de
    % la colonne de gauche, grâce notamment au fait que le bloc de gauche a été
    % explicitement réglé à la même hauteur que les colonnes.
    % [EN] This \vfill allows the following block to be pushed to the bottom of
    % the column, hence making the bottom of the block aligned with the bottom
    % of the left column block, thanks to the fact that the left block was set
    % to have the same height as the columns.
    \vfill

    \lirisblock{Bibliographie}{
        \printbibliography
    }
}

\end{frame}
\end{document}
